\section{Design goals}
\begin{itemize}
\item Strong and easy usability\\
The goal with the systems usability, is to end up with a product which is user-friendly for all types of users. The gestalt laws are excellent examples on how to make a clean and simple user interface. The gestalt laws focuses on the placement of different GUI objects, such as buttons and text fields in a user interface, to make the interaction between user and user interface easy and painless for the user. By using the gestalt laws, we are able to make a clear distinction of which parts of the user interface that belongs to each other and thereby make it easier for the user to interact with the user interface. This will also mean that even the unexperienced user will have no trouble with interacting with our user interface.\\

\item High Reliabiltiy\\
It is desired to create a reliable program that will prevent the user from losing progress made to the system, due to unexpected events, whereas an event could be a computer turn-off by accident or any other such similar things. How our system will be handling this, is by doing frequent auto-saving of the data, to a local data file, and then synchronize the auto-saved data with our data storage, to have an external backup of the data. Furthermore, every time the user commits anything, whereas this for example could be a new calendar entry, it will also be saved immediately to the local data file, and thereafter synchronize the local data file with the database. Force restarting should not be acceptable and most exceptions must be caught and handled during runtime. By doing the abovementioned, data loss will be kept to a minimum by limiting it to the user’s current activity, should a failure occur.\\

\item Solid Performance\\
The goal performance-wise is to be able to handle a large number of events daily, without setting a noticeable strain on the program. Heavy operations must run in the background, and must therefore not disturb the user while operating the program. The trade-off, however, will be the loading time of the program. This allows us to load all the initially required data, and prevent long waiting times while operating the system. The Calendar is a lightweight system, and stores data on the cloud, so space wise, a maximum of around 1gb should be achievable. Additionally, in its current form the CalendarSystem is only runnable on Windows OS.\\

\item Strong achritecture with focus on extendability\\
When rolling out future updates for the system, a full re-installation of the program will be necessary. This is due to resources allocated to other more desired design goals.\\

\item Good documentation and Testing of the most important subsystems\\
The system will be tested before release, but in a limited way. As a full system test can’t be accomplished due to the time frame set for this systems development, a fully thorough testing will not be attainable, and we will therefore only focus on testing key components that are central parts of our system.
Documentation will be a central part of our system, as it is important to keep documentation of how the system works as a whole, and how the modules or subsystem works separately, and by making this documentation, it will be easier to extend our system by developers that have no knowledge about the system\\
\end{itemize}