\section{Overview}
\textbf{View subsystem:}
This subsystem contains the different types of views that will show, depending on how the user is 
using the system. Therefore, its assignment of functionality will be that this subsystem 
has to keep track of the different views.\\
\textbf{Model subsystem:}
This subsystem contains the data that the system will be using. When the model changes state, 
it will notify the controllers, and the controllers will then change the view. Therefore, 
this subsystems assignment of functionality will be notifying the controllers, 
when its state is being changed.\\
\textbf{Controller subsystem:}
This subsystem is being used to update the views, by receiving the models state. 
Its assignment of functionality will be that this subsystem will have to update the views 
with the received models state.\\
\textbf{DataStorage subsystem:}
The DataStorage subsystem creates persistance of data either through a Database connection or locally in a file storage. The Client can gain access to the storage through a storage client which controls which kind of storage must be used at a given time and also controls the logic for the model to storage convertion. 

\subsection{UML class Diagram}
Along the project comes a UML Class Diagram file which can be viewed in Visual studio. Here we can see the different subsystems in action. Topmost is the Storage subsystem. The subsystem is in the form of a Abstract Factory pattern. The IAbsatract storage has a factory which creates Abstractstorages which has IStorage. \\In this section the bridge pattern is also in use (the abstract storage works as a bridge) and the strategy pattern in the form of the Offline and Online connection classes.\\

Beneath the storage subsystem we can see the view subsystem. The classes all share the IViews interface which provides the classe with basic view methods such as show, hide and clear.\\

At the bottum of the UML class diagram we can see the Events and how they are implemented. The Event section of the model subsystem uses a couple of design patterns to increase coherence and loosen the coupling. The GoogleCalendarEvent is the result of the Adapter pattern and The LinkedEvents class is a result of the Composite pattern.